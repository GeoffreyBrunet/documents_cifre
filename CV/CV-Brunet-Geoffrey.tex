\documentclass[a4paper,twoside]{article}
\usepackage[utf8]{inputenc}
\usepackage{geometry}
\geometry{margin=2cm}
\usepackage{fancyhdr}
\usepackage{multicol}
\usepackage[T1]{fontenc}
\usepackage{lmodern}
\usepackage{hyperref}

\pagestyle{fancy}
\fancyhead{}
\fancyfoot{}
\fancyhead[RO]{\leftmark}
\fancyhead[LE]{\rightmark}

\begin{document}

\begin{multicols}{2}

    \section*{BRUNET GEOFFREY}

    \noindent 10 rue du Grand Caire, Auxerre, 89000\ \newline
    06.23.35.49.26 | \href{mailto:geoffrey-brunet89@outlook.fr}{geoffrey.brunet@icloud.com}\
    \href{https://www.linkedin.com/in/geoffrey-brunet-558315ba/}{Linkedin}: \url{https://www.linkedin.com/in/geoffrey-brunet-558315ba/}\\
    \href{https://github.com/GeoffreyBrunet}{Github}: \url{https://github.com/GeoffreyBrunet}

    \section*{OBJECTIF}

    \noindent En dernière année de bac +5, je souhaite continuer sur un Doctorat, en intelligence artificielle.

    \section*{EXPÉRIENCE}

    \begin{itemize}
        \item Alternance depuis septembre 2020 chez Quartz Insight en tant que développeur
              \begin{itemize}
                  \item Création d'un outil de gestion de licences pour le logiciel de l'entreprise.
                  \item Création d'un outil d'acquisition et de gestion de journaux d'évènements.
                  \item Création d'une API REST en java pour gérer et executer des taches planifiées.
              \end{itemize}
        \item Alternance de 2 ans dans chez Louis 21 d’aout 2016 à Juillet 2018.
        \item Entreprise Netquarks à Paris en mai/juin 2015 et décembre 2015/février 2016, en stage puis en emploi saisonnier.
    \end{itemize}

    \section*{DIPLÔMES}
    \begin{itemize}
        \item Bachelor Concepteur / Développeur d’applications
        \item BTS SIO option Infrastructure Systèmes et Réseaux, en 2018 (major de promotion).
        \item Bac professionnel systèmes électroniques numériques, en 2016, mention ''bien''.
        \item Bac professionnel laboratoire contrôle qualité, en 2014, mention ''assez bien''.
        \item Bac professionnel laboratoire contrôle qualité, en 2014, mention ''assez bien''.
        \item Brevet d'études professionnelles travaux de laboratoire, en 2013.
        \item Brevet des collèges, en 2009.
    \end{itemize}

    \section*{CONNAISSANCES
      \newline TECHNIQUES}

    \begin{itemize}
        \item Langages de programmation: Python, C++, Java
        \item Éditeurs de texte \& IDEs: VSCode, Jupyter-lab (\& Google Colab), PyCharm
        \item Outils de science des données: Numpy, Pandas, Scipy, Sympy
        \item Machine learning: PyTorch, Scikit-Learn, XGBoost
        \item Graphiques \& presentations: Matplotlib, Streamlit
        \item Bases de données: PostgreSQL, SQLite, Redis
        \item Management et gestion des versions: Git, Anaconda, Docker, Gitlab-CI
        \item Web APIs: FastAPI
        \item Outils de configurations et build: Hatch, CMake
        \item Autre: LaTeX
    \end{itemize}

    \section*{CENTRES D'INTERÊT}

    \begin{itemize}
        \item Informatique DIY et impression 3D
        \item Voyages
        \item Running / Trail (Tout terrains)
        \item Photographie argentique et numérique / cinéma
    \end{itemize}

\end{multicols}
\end{document}
